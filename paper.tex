\documentclass[runningheads]{llncs}

\usepackage[dvipsnames]{xcolor}
\usepackage{complexity}
\usepackage{graphicx}
\usepackage{rotating}
\usepackage{tikz}
\usetikzlibrary{trees, positioning, calc, shapes}
\usepackage{fancyvrb}
\usepackage{amssymb}
\usepackage{amsmath}
\usepackage{cleveref}
\usepackage{nicefrac}

% \usepackage{showframe}

% cref style
\crefname{figure}{Figure}{Figures}
\Crefname{figure}{Figure}{Figures}
\crefname{section}{Section}{Sections}
\Crefname{section}{Section}{Sections}

\title{VeriPB: The Easy Way to Make Your Combinatorial Search Algorithm Trustworthy}
\titlerunning{VeriPB: Make Your Combinatorial Search Algorithm Trustworthy}

\author{
    Stephan Gocht\inst{1,2}%\orcidID{0000-0002-5459-3134}
    \and Ciaran McCreesh\inst{3}%\orcidID{0000-0002-6106-4871}
    \and Jakob Nordstr\"{o}m\inst{2,1}%\orcidID{0000-0002-2700-4285}
}

\institute{
    Lund University, Lund, Sweden \and
    University of Copenhagen, Copenhagen, Denmark \and
    University of Glasgow, Glasgow, Scotland \\
\email{stephan.gocht@cs.lth.se, ciaran.mccreesh@glasgow.ac.uk, jn@di.ku.dk}}

\begin{document}

\maketitle

\begin{abstract}
    We give an overview of the VeriPB tool, which can be used to make certifying solvers for
    combinatorial optimisation problems.
\end{abstract}

The VeriPB tool\footnote{\url{https://github.com/StephanGocht/VeriPB}} provides an easy way of
making combinatorial search algorithms \emph{certifying}. Certifying solvers, which output a proof
of correctness alongside a claimed solution, are now standard in the Boolean satisfiability (SAT)
community, but have not seen much uptake elsewhere. The key problem with the SAT approach to proof
logging is that their chosen proof formats, CNF and DRAT
\cite{DBLP:conf/sat/WetzlerHH14}, make it
extremely impractical to describe the more sophisticated reasoning used in constraint programming
propagators \cite{DBLP:conf/aaai/ElffersGMN20} and in other combinatorial search algorithms
\cite{GochtMMNPT20,DBLP:conf/ijcai/GochtMN20}. In contrast, the psuedo-Boolean models and
cutting planes proof system used by VeriPB makes it easy to justify combinatorial arguments. This
tool demonstration gives a solver author's perspective of using VeriPB to make an algorithm
implementation trustworthy.

When a solver claims than an instance of an NP-complete problem is unsatisfiable, it is hard to be
confident that the solver is correct. Similarly, for optimisation problems, it is hard to be sure
that a claimed solution is indeed optimal, and for enumeration problems, it is hard to be sure that
no solutions have been missed. The VeriPB tool handles all of these scenarios through \emph{proof
logging}, which is a particular form of \emph{certifying} \cite{DBLP:journals/csr/McConnellMNS11}.
The key steps involved are:

\begin{itemize}
    \item The problem is expressed as a pseudo-Boolean (PB) problem instance, encoded in the
        standard OPB file format \cite{opbformat}.
        This can either be done by the solver, or in the case of solver competitions, can be
        provided by the competition organiser. A PB instance is simply an integer linear program
        where all the variables have domain $\{ 0, 1 \}$; conveniently, CNF clauses can be
        represented directly as PB constraints.
    \item As it executes, the solver outputs a machine-readable proof log which records the steps it
        took to reach its solution. This proof log consists of \emph{reverse unit propagation} (RUP)
        steps \cite{DBLP:conf/aaai/ElffersGMN20,DBLP:conf/isaim/Gelder08} which record every time the solver
        backtracks, together with additional manual constraint derivations using \emph{cutting
        planes} proof system rules \cite{DBLP:journals/dam/CookCT87} to justify any
        complex constraint propagation or bounds; these manual derivations are used to ensure that
        all inferences performed by the solver can be reflected inside the verifier just through using
        unit propagation. For satisfiable, enumeration, and optimisation
        instances, this log also records solutions or new incumbents as they are found.
    \item A proof verifier such as VeriPB takes the OPB file and the proof log, and checks whether
        the proof is valid. A proof verifier is a very simple piece of software when compared to a
        typical solver, and so it is much easier to trust. In particular, the proof verifier can
        only carry out very simple inference steps and only as directed, and does not perform any search.
\end{itemize}

From the point of view of end users and solver authors, this approach has a number of advantages
compared to full formal verification \cite{Dubois20}, and compared to earlier proof logging approaches
\cite{Stuckey19,DBLP:conf/aaai/VekslerS10,DBLP:conf/sat/WetzlerHH14}, which
together mean we now believe it is practical to pursue proof logging as the new ``socially
acceptable standard'' for solver implementers.

Firstly, proof logs can be stored, and audited at a later date. It is even reasonably simple to
implement an independent verifier, because the verifier does not need to understand constraint
propagation or bounds. For example, because all-different reasoning can be justified compactly using
cutting planes steps, the verifier can verify Hall set reasoning \cite{DBLP:conf/aaai/ElffersGMN20},
without knowing what a Hall set is or how all-different propagation works. Similarly, the wide
variety of bound and inference functions used in modern subgraph-finding algorithms
\cite{GochtMMNPT20,DBLP:conf/ijcai/GochtMN20} can all be verified without the verifier having any
knowledge of graphs.

Secondly, this form of proof logging is simple to implement inside solvers. For all of the examples
we have tried so far \cite{DBLP:conf/aaai/ElffersGMN20,GochtMMNPT20,DBLP:conf/ijcai/GochtMN20}, it
has taken \emph{much} less time (between a factor of two and a factor of many hundreds) to implement
proof logging in an existing solver than it took to implement the solver itself. This remained true
even for non-experts. The key to this is RUP~\cite{DBLP:conf/aaai/ElffersGMN20}: for
search, the solver needs only output the trail
every time it backtracks. Meanwhile, most constraint propagation steps do not require explicit proof
derivations: RUP ensures that any inference steps which follow by integer bounds consistency
\cite{DBLP:conf/ausai/ChoiHLS06} from the provided PB model (which for clausal constraints, is the
same as unit propagation in SAT, but in general is more powerful) are handled implicitly. For more
complex propagators or bounds functions, their behaviour must be justified, but this can be done
without needing to consider the trail; furthermore, it is often easy to reuse justification
templates between different algorithms and solvers.

Thirdly, the process is at least reasonably efficient. A particular goal has been to ensure that
proof logs are, in some sense, not longer than the amount of work carried out by a solver. The
cutting planes proof system appears to be particularly suitable here, being able to express a wide
range of combinatorial arguments
\cite{DBLP:conf/aaai/ElffersGMN20,GochtMMNPT20,DBLP:conf/ijcai/GochtMN20}. This would not be the
case if we used, for example, the resolution proof system, which would require an exponential blowup
for justifying Hall set reasoning \cite{DBLP:journals/tcs/Haken85,Stuckey19}.

Fourthly, the process certifies \emph{solutions}, rather than proving a solver correct. This is a
mixed blessing. It does not guarantee that a solver will never produce an incorrect answer, but it
does ensure that if an incorrect answer is ever produced (or if a correct answer is produced using
unsound reasoning), then it can be detected. This holds even if the error was due to a compiler bug
or a hardware fault, or due to the solver relying upon an algorithm whose purported proof of
correctness turned out to be spurious.

And finally, we have found that proof logging can help with solver development, catching bugs early
during the implementation process that conventional testing had missed. VeriPB includes a number of
features to help with this, such as the ability to trace proof logs as they are executed, the
ability to assert that derived constraints imply expected consequences, and the ability
to instruct the verifier to accept certain facts on faith, to enable incremental
development.

\bibliographystyle{splncs04}
\bibliography{paper}

\end{document}

% vim: set tw=100 spell spelllang=en : %
